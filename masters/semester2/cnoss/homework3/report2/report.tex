\documentclass{vilniustech-en}
\vilniustechsetup{
    university={Vilnius Gediminas technical university},
    faculty={Faculty of Fundamental Sciences},
    cathedral={Department of Information Systems},
    workTitle={Computer Networks and Operating System Security},
    workType={IDS/IPS},
    workAuthorName={Aurimas Šakalys},
    workAuthorGroup={ITSfm-22},
    workRecipient={lect. Vitalijus Gurčinas}
}
\addbibresource{bibliography.bib}
\VTDocumentBegin

\section{Infrastructure}

We are reusing the setup we had for the firewall laboratory work (\autoref{fig:topology}). In this case, we remove all the firewall rules we had previously created for ease of use. Additionally, we will only be using external and firewall machines. 

\VTImage
{img/topology.png}
{Topology of our network example}
{fig:topology}
{16cm}

In the firewall machine, we will install and configure \textit{Suricata} and \textit{ELK} stack, displaying the logs/alerts generated by \textit{Suricata}.

We set both internal and external IP addresses to the same network ID (\autoref{fig:ids_variables_suricata}), as we will only monitor traffic on the external interface.

\VTImage
{img/ids_variables_suricata.png}
{\textit{Suricata} interface configuration}
{fig:ids_variables_suricata}
{16cm}

We will skip showing the configuration for the \textit{ELK} stack itself, as it is irrelevant to the task.

\section{Rules}

\subsection{\textit{ICMP}}

\subsubsection{\textit{DoS}}

We assume there is a possibility of a \textit{DoS} attack on our infrastructure. We will create a rule that will alert us of an attack known as an \textit{ICMP} flood (\autoref{lst:suricata_icmp_flood}) \footnote{The rules in the listings are missing the \textbackslash\ symbol at the end of the lines. They were removed for clarity. They are required if we want to use these rules as is in our \textit{Suricata} rulesets.}. In this attack, one host sends out a barrage of \textit{ICMP} packets to overwhelm the server.

\begin{lstlisting}[language={json}, caption={\textit{ICMP} flood \textit{Suricata} rule}, label={lst:suricata_icmp_flood}, captionpos=b]
alert icmp any any -> $HOME_NET any
(
    msg:"Detected ICMP Flood";
    threshold: type both, track by_src, count 100, seconds 5;
    sid:1000001;
    rev:1;
)
\end{lstlisting}

The rule accumulates \textit{ICMP} packet requests to our internal network by the packet's origin. As such, if we receive a couple of \textit{ICMP} requests from many hosts, this rule would not trigger.

To detect an \textit{ICMP} \textit{DDoS} attack, we would need to create an additional rule to accumulate packets by the destination and adjust the count and time threshold accordingly.

\subsubsection{Testing \textit{ICMP} \textit{DoS} rule}

To test this rule, we will run the \textit{ping -f 192.168.56.1} command to flood the server with \textit{ICMP} packets (\autoref{fig:ids_icmp_flood_command}). As we can see, we have sent 55021 packets within 8 seconds, meaning an alert should have been raised and be visible within \textit{Elastic}.

\VTImage
{img/ids_icmp_flood_command.png}
{\textit{ICMP} packet flood}
{fig:ids_icmp_flood_command}
{16cm}

Unfortunately, no alert has been raised (\autoref{fig:ids_icmp_flood_no_result}). We have tried to do multiple iterations of this rule, using \textit{threshold}, \textit{detection\_filter} and a lot of different configurations to make this rule work. We have tried an elementary rule to raise an alert on any \textit{ICMP} packets coming in. These incoming packets are being processed correctly. The thresholding rule should technically work, but it does not. The issue is likely in the configuration of Suricata rather than the rule's configuration \footnote{We create another rule using a very similar configuration in the future, and it works correctly.}. 

\VTImage
{img/ids_icmp_flood_no_result.png}
{No results for \textit{ICMP} packet flood attack}
{fig:ids_icmp_flood_no_result}
{16cm}

\subsubsection{Ping of Death}
The Ping of Death attack can occur when an \textit{ICMP} packet exceeding the maximum size of the \textit{IP} packet size is sent. Due to some implementation issues, this packet might cause the server to malfunction. 

The rule definition (\autoref{lst:suricata_pod_rule}) might seem counter-intuitive - the maximum size of an \textit{IP} packet is 65535 bytes, and we are only checking for a size larger than 1400 bytes. In regular use, any legitimate \textit{ICMP} packets should be at most the size of 1400 bytes; as such, our rule is reasonable.

\begin{lstlisting}[language={json}, caption={Ping of Death \textit{Suricata} rule}, label={lst:suricata_pod_rule}, captionpos=b]
alert icmp any any -> $HOME_NET any
(
    msg:"Potential Ping of Death";
    dsize:>1400;
    sid:1000009;
    rev:1;
)
\end{lstlisting}

\subsubsection{Testing Ping Of Death rule}

To test this rule, we use the command \textit{ping -s 1500 192.168.56.1}, which sends out \textit{ICMP} packets with a size of 1500 bytes (\autoref{fig:ids_ping_big_ping}).

\VTImage
{img/ids_ping_big_ping.png}
{Sending large packets to the server}
{fig:ids_ping_big_ping}
{16cm}

With this test complete, we observe the changes in \textit{Elastic}. We can see that the rule was triggered, and the alert was created (\autoref{fig:ids_ping_big_works}).

\VTImage
{img/ids_ping_big_works.png}
{Created alerts for \textit{ICMP} Ping of Death attack}
{fig:ids_ping_big_works}
{16cm}

\subsection{Outgoing \textit{SSH} traffic}

We shall assume that the security policy for the server states that \textit{SSH} connections should only be initiated to the server. Any connections originating from the server should be checked and generate an alert (\autoref{fig:suricata_ssh_init}).

\begin{lstlisting}[language={json}, caption={\textit{SSH} traffic initialization \textit{Suricata} rule}, label={lst:suricata_ssh_init}, captionpos=b]
alert tcp $HOME_NET any -> any 22
(
        msg:"Detected initialisation of outgoing SSH traffic";
        flow:to_server,established;
        content:"SSH-";
        offset:0;
        depth:4;
        sid:1000002;
        rev:1;
)
\end{lstlisting}

\subsubsection{Testing outgoing \textit{SSH} traffic rule}

We shall connect to the external machine via \textit{SSH} to test this rule (\autoref{fig:ids_ssh_connect}).

\VTImage
{img/ids_ssh_connect.png}
{Connecting to the external machine via \textit{SSH}}
{fig:ids_ssh_connect}
{16cm}

With this test complete, we observe the changes in \textit{Elastic}. We can see that the rule was triggered, and the alert was created (\autoref{fig:ids_ssh_hit}).

\VTImage
{img/ids_ssh_hit.png}
{Created alerts for outgoing \textit{SSH} traffic rule}
{fig:ids_ssh_hit}
{16cm}

\subsection{\textit{FTP} rules}
\subsubsection{Anonymous \textit{FTP} login}
The following rule (\autoref{lst:suricata_ftp_anonymous}) allows us to detect any anonymous connection from the server. We check the start of the packet for data indicating that an anonymous login has been attempted.

\begin{lstlisting}[language={json}, caption={\textit{FTP} anonymous login \textit{Suricata} rule}, label={lst:suricata_ftp_anonymous}, captionpos=b]
alert tcp $HOME_NET any -> any 21 
( 
    msg:"FTP anonymous login"; 
    flow:established,to_server; 
    content:"USER"; 
    nocase; 
    content:"anonymous"; 
    within:10; 
    sid:1000003; 
    rev:1; 
)
\end{lstlisting}

\subsubsection{\textit{FTP} file upload}
The following rule (\autoref{lst:suricata_ftp_upload}) allows us to detect an attempt to upload a file to an \textit{FTP} server. We check for the \textit{STOR} command to the \textit{FTP} server as this command initiates the \textit{FTP} file uploads.

\begin{lstlisting}[language={json}, caption={\textit{FTP} file upload \textit{Suricata} rule}, label={lst:suricata_ftp_upload}, captionpos=b]
alert tcp $HOME_NET any -> any 21
(
    msg:"FTP File Upload Detected";
    flow:established,to_server;
    content:"STOR";
    sid:1000010;
    rev:1;
)
\end{lstlisting}

\subsubsection{Testing \textit{FTP} rules}

We will test both of these rules at the same time. We run a simple \textit{FTP} server on the external machine with anonymous logins enabled (\autoref{fig:ids_ftp_server}). 

\VTImage
{img/ids_ftp_server.png}
{Enabling \textit{FTP} server with anonymous logins}
{fig:ids_ftp_server}
{16cm}

Then we try to connect to it from our server and attempt to upload a local file (\autoref{fig:ids_ftp_connect_and_upload}).

\VTImage
{img/ids_ftp_connect_and_upload.png}
{Anonymous login to the \textit{FTP} server and attempted file upload}
{fig:ids_ftp_connect_and_upload}
{16cm}

Even though our file upload routine failed, Suricata raised an alert for both situations (\autoref{fig:ids_ftp_hit}).

\VTImage
{img/ids_ftp_hit.png}
{Created alerts for \textit{FTP} rules}
{fig:ids_ftp_hit}
{16cm}

\subsection{\textit{HTTP} rules}

The following rules are designed to allow us to notice suspicious \textit{HTTP} traffic. 

The first rule checks for unconventional requests. We assume that we expect only regular web \textit{HTTP} traffic in our environment. As such, a request with a \textit{curl} user agent would be considered irregular traffic (\autoref{lst:suricata_http_curl}).

\begin{lstlisting}[language={json}, caption={\textit{HTTP} \textit{curl} user agent \textit{Suricata} rule}, label={lst:suricata_http_curl}, captionpos=b]
alert http any any -> $HOME_NET any
(
    msg:"Incoming HTTP request with curl User-Agent";
    flow:established,to_server;
    content:"User-Agent:|20|curl";
    http_header;
    nocase;
    sid:1000004;
    rev:1;
)
\end{lstlisting}

The second rule checks for fetching suspicious files from the server (\autoref{lst:suricata_http_sh}). This rule will alert us if malicious scripts are downloaded from the web. 

\begin{lstlisting}[language={json}, caption={\textit{HTTP} \textit{.sh} file download \textit{Suricata} rule}, label={lst:suricata_http_sh}, captionpos=b]
alert http $HOME_NET any -> any any
(
    msg:"Outgoing HTTP request to .sh file";
    flow:established,to_client;
    content:".sh";
    http_uri;
    sid:1000005;
    rev:1;
)
\end{lstlisting}

The third rule checks for an attempted SQL injection attack. It tries to match \textit{' or '1'='1} (\textit{URI} encoded) within the \textit{URI} (\autoref{lst:suricata_http_sql}). This rule is one of the more common ways to perform a test for \textit{SQL} injection vulnerabilities.


\begin{lstlisting}[language={json}, caption={\textit{HTTP} \textit{SQL} injection \textit{Suricata} rule}, label={lst:suricata_http_sql}, captionpos=b]
alert http any any -> $HOME_NET any
(
    msg:"SQL Injection attempt";
    flow:established,to_server;
    content:"%27%20or%20%271%27%3D%271";
    http_uri;
    nocase;
    sid:1000006;
    rev:1;
)
\end{lstlisting}

\subsubsection{Testing \textit{HTTP} rules with server origin}

We will test both the \textit{curl} and \textit{SQL} injection attack discovery at the same time. To do so, we launch a simple HTTP server on the server (\autoref{fig:ids_http_server_up}).

\VTImage
{img/ids_http_server_up.png}
{Running \textit{HTTP} server on the server machine}
{fig:ids_http_server_up}
{16cm}

Once the HTTP server is up, we craft a targeted `curl` request, which includes the SQL injection attack (\autoref{fig:ids_http_server_test}). This request should trigger both of these rules.

\VTImage
{img/ids_http_server_test.png}
{\textit{curl} request with specially crafted \textit{URI}}
{fig:ids_http_server_test}
{16cm}

With this test complete, we observe the changes in Elastic. We can see that both rules were triggered, and the alerts were created (\autoref{fig:ids_http_rules_trigger}).

\VTImage
{img/ids_http_rules_trigger.png}
{Created alerts for \textit{HTTP} rules}
{fig:ids_http_rules_trigger}
{16cm}

\subsubsection{Testing \textit{HTTP} rules with external origin}
To test the script downloading rule, we set up the simple \textit{HTTP} server on the external machine and created an empty script with the \textit{.sh} extension (\autoref{fig:ids_http_bad_script_up}).


\VTImage
{img/ids_http_bad_script_up.png}
{Created alerts for \textit{HTTP} rules}
{fig:ids_http_bad_script_up}
{16cm}

Once the server is running on the external machine, we use \textit{wget} to download the script (\autoref{fig:ids_http_bad_script_wget}).

\VTImage
{img/ids_http_bad_script_wget.png}
{Fetching the empty script from the external machine}
{fig:ids_http_bad_script_wget}
{16cm}

With this test complete, we observe the changes in \textit{Elastic}. We can see that the script downloading rule was triggered, and the alert was created (\autoref{fig:ids_elastic_bad_script}).

\VTImage
{img/ids_elastic_bad_script.png}
{Created alerts for \textit{HTTP} rule}
{fig:ids_elastic_bad_script}
{16cm}

\subsection{\textit{DNS} rules}

This rule triggers an alert for \textit{DNS} queries for the \textit{example.com} domain. By setting up rules for \textit{DNS}, we could get alerts when requests for weird domain names appear.

\begin{lstlisting}[language={json}, caption={\textit{DNS} resolution \textit{Suricata} rule}, label={lst:suricata_dns}, captionpos=b]
alert dns any any -> any 53 
( 
    msg:"DNS request for example.com"; 
    dns_query; 
    content:"example.com"; 
    nocase; 
    sid:1000007; 
    rev:1; 
)
\end{lstlisting}

\subsubsection{Testing \textit{DNS} rules}
To test the \textit{DNS} resolution rule, we will simply run the \textit{dig} command to resolve the \textit{DNS} query (\autoref{fig:ids_dig_through_external}). We have to specify that we want to resolve this through the external machine, as it goes to the \textit{NAT} interface by default.

\VTImage
{img/ids_dig_through_external.png}
{\textit{DNS} resolution request via the external machine}
{fig:ids_dig_through_external}
{16cm}

With this test complete, we observe the changes in \textit{Elastic}. We can see that the \textit{DNS} resolution rule was triggered, and the alert was created (\autoref{fig:ids_dig_trigger}).

\VTImage
{img/ids_dig_trigger.png}
{Created alerts for \textit{DNS} rule}
{fig:ids_dig_trigger}
{16cm}

\subsection{Mail rule}

This rule has been designed to alert us if any emails with attachments are going out from the server \footnote{In reality this traffic would be forwarded through the \textit{IDS}}, indicating possible data exfiltration attempts (\autoref{lst:suricata_email}).

\begin{lstlisting}[language={json}, caption={Email with attachment \textit{Suricata} rule}, label={lst:suricata_email}, captionpos=b]
alert tcp $HOME_NET any -> any 25
(
    msg:"Outgoing SMTP with attachment";
    flow:established,to_server;
    content:"Content-Disposition:|20|attachment";
    nocase;
    sid:1000008;
    rev:1;
)
\end{lstlisting}

\subsubsection{Testing mail rule}
To test the outgoing \textit{SMTP} rule, we run a simple \textit{SMTP} server using \textit{Python} on the external machine and send an email with an attachment from the server (\autoref{fig:ids_smtp_server_better}).

\VTImage
{img/ids_smtp_server_better.png}
{Running the \textit{SMTP} server on the external machine}
{fig:ids_smtp_server_better}
{16cm}

We use the \textit{swaks} to send the email with an attachment to the specified \textit{SMTP} server (\autoref{fig:ids_email_with_attachment_send}). 

\VTImage
{img/ids_email_with_attachment_send.png}
{Sending the email with attachment}
{fig:ids_email_with_attachment_send}
{16cm}

With this test complete, we observe the changes in \textit{Elastic}. We can see that the \textit{SMTP} with the attachment rule was triggered, and the alert was created (\autoref{fig:ids_smtp_attachment_hit}).

\VTImage
{img/ids_smtp_attachment_hit.png}
{Created alerts for email rule}
{fig:ids_smtp_attachment_hit}
{16cm}

\subsection{\textit{DoS} rule}
The following rule alerts us if a substantial amount of \textit{SYN} packets are received, which might indicate an \textit{SYN} flood \textit{DoS} attack (\autoref{lst:suricata_dos}). \footnote{This rule is very similar to the rule to detect \textit{ICMP} flood, but as we will see, it does manage to trigger. As this rule works correctly, this indicates some issues with the thresholding logic for \textit{ICMP} protocol in \textit{Suricata}.}

\begin{lstlisting}[language={json}, caption={\textit{DoS} attack \textit{Suricata} rule}, label={lst:suricata_dos}, captionpos=b]
alert tcp any any -> $HOME_NET any
(
    msg:"Potential DoS attack";
    flags: S;
    threshold: type both, track by_src, count 100, seconds 1;
    sid:1000011;
    rev:1;
)
\end{lstlisting}

\subsubsection{Testing \textit{DoS} rule}

To test this out, we use \textit{hping3} to send out many \textit{SYN} packets in a very short time (\autoref{fig:ids_ddos_syn_flood}).

\VTImage
{img/ids_ddos_syn_flood.png}
{Using \textit{hping3} to send out many \textit{SYN} packets}
{fig:ids_ddos_syn_flood}
{16cm}

With this test complete, we observe the changes in \textit{Elastic}. We can see that the \textit{DoS} \textit{SYN} flood attack rule was triggered, and the alert was created (\autoref{fig:ids_ddos_syn_triggered}).

\VTImage
{img/ids_ddos_syn_triggered.png}
{Created alerts for \textit{DoS} rule}
{fig:ids_ddos_syn_triggered}
{16cm}

This is an additional \textit{DoS} rule created per the requirements, the \textit{ICMP} flood rule also does fit the requirement for a \textit{DoS} rule.

\subsection{Protocol Anomaly rule}

A rule that alerts if there is data in the \textit{SYN} packet (\autoref{fig:suricata_protocol_anomaly}). By default, an \textit{SYN} packet should have no data; it might indicate some scanning attack or an attack on the protocol implementation itself.

\begin{lstlisting}[language={json}, caption={Protocol anomaly \textit{Suricata} rule}, label={lst:suricata_protocol_anomaly}, captionpos=b]
alert tcp any any -> $HOME_NET 80
(
    msg:"TCP protocol anomaly - data in SYN packet";
    flow:stateless;
    flags: S;
    dsize:>0;
    sid:1000012;
    rev:1;
)
\end{lstlisting}

\subsubsection{Testing Protocol Anomaly rule}

To test, we use \textit{scapy} \textit{Python} library to craft a \textit{TCP/IP} \textit{SYN} packet that contains data (\autoref{fig:ids_protocol_anomaly_test}).

\VTImage
{img/ids_protocol_anomaly_test.png}
{\textit{scapy} script to send a malformed \textit{SYN} packet}
{fig:ids_protocol_anomaly_test}
{16cm}

With this test complete, we observe the changes in \textit{Elastic}. We can see that the protocol anomaly rule was triggered, and the alert was created (\autoref{fig:ids_protocol_anomaly_hit}).

\VTImage
{img/ids_protocol_anomaly_hit.png}
{Created alerts for protocol anomaly rule}
{fig:ids_protocol_anomaly_hit}
{16cm}

\subsection{Web Application Rule}
This rule alerts us if there is a possible \textit{XSS} attack via the \textit{URI}. We check for the \textit{<script>} tag in the \textit{URI} (\autoref{lst:suricata_xss}).


\begin{lstlisting}[language={json}, caption={\textit{XSS} attack \textit{Suricata} rule}, label={lst:suricata_xss}, captionpos=b]
alert http any any -> $HOME_NET any
(
    msg:"Potential Web Application Attack - XSS";
    flow:established,to_server;
    content:"<script>";
    http_uri;
    nocase;
    sid:1000013;
    rev:1;
)
\end{lstlisting}

\subsubsection{Testing Web Application rule}
To test this, we craft a specific \textit{URI} that would include the script tag in the \textit{URI} and use \textit{curl} to send this request (\autoref{fig:ids_xss_test}).

\VTImage
{img/ids_xss_test.png}
{Sending pretend \textit{XSS} traffic via \textit{curl}}
{fig:ids_xss_test}
{16cm}

With this test complete, we observe the changes in \textit{Elastic}. We can see that the \textit{XSS} and \textit{curl} rules were triggered, and alerts were created (\autoref{fig:ids_xss_hit}).

\VTImage
{img/ids_xss_hit.png}
{Created alerts for \textit{XSS} rule}
{fig:ids_xss_hit}
{16cm}

This is an additional web application rule created per the requirements, the \textit{SQL} injection rule also does fit the requirement for a web application rule.

\section{Conclusion}
In this exercise, we conducted a few security tests using \textit{Suricata} and \textit{ELK} for log management. We created rules to detect many different types of attacks or anomalous traffic. Despite a few hiccups on some rules, most were created correctly and tested.

\VTDocumentEnd
